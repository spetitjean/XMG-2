%%%%%%%%%%%%%%%%%%%%%%%%%%%%%%%%%%%%%%%%%
% The Legrand Orange Book
% LaTeX Template
% Version 1.0 (25/3/13)
%
% This template has been downloaded from:
% http://www.LaTeXTemplates.com
%
% Original author:
% Mathias Legrand (legrand.mathias@gmail.com)
%
% License:
% CC BY-NC-SA 3.0 (http://creativecommons.org/licenses/by-nc-sa/3.0/)
%
% Compiling this template:
% This template uses biber for its bibliography and makeindex for its index.
% This means that to update the bibliography and index in this template you
% will need to run the following sequence of commands in the template
% directory:
%
% 1) pdflatex main
% 2) makeindex main.idx -s StyleInd.ist
% 3) biber main
% 4) pdflatex main
%
% This template also uses a number of packages which may need to be
% updated to the newest versions for the template to compile. It is strongly
% recommended you update your LaTeX distribution if you have any
% compilation errors.
%
% Important note:
% Chapter heading images should have a 2:1 width:height ratio,
% e.g. 920px width and 460px height.
%
%%%%%%%%%%%%%%%%%%%%%%%%%%%%%%%%%%%%%%%%%

%----------------------------------------------------------------------------------------
%	PACKAGES AND OTHER DOCUMENT CONFIGURATIONS
%----------------------------------------------------------------------------------------

\documentclass[11pt,fleqn]{book} % Default font size and left-justified equations

\usepackage[top=3cm,bottom=3cm,left=3.2cm,right=3.2cm,headsep=10pt,a4paper]{geometry} % Page margins

\usepackage{xcolor} % Required for specifying colors by name
\definecolor{ocre}{RGB}{243,102,25} % Define the orange color used for highlighting throughout the book

% Font Settings
\usepackage{avant} % Use the Avantgarde font for headings
%\usepackage{times} % Use the Times font for headings
\usepackage{mathptmx} % Use the Adobe Times Roman as the default text font together with math symbols from the Sym­bol, Chancery and Com­puter Modern fonts
\usepackage{microtype} % Slightly tweak font spacing for aesthetics
\usepackage[utf8]{inputenc} % Required for including letters with accents
\usepackage[T1]{fontenc} % Use 8-bit encoding that has 256 glyphs

% Bibliography
\usepackage[style=alphabetic,sorting=nyt,sortcites=true,autopunct=true,babel=hyphen,hyperref=true,abbreviate=false,backref=true,backend=biber]{biblatex}
%\addbibresource{bibliography.bib} % BibTeX bibliography file
\defbibheading{bibempty}{}

% Index
\usepackage{calc} % For simpler calculation - used for spacing the index letter headings correctly
\usepackage{makeidx} % Required to make an index
\makeindex % Tells LaTeX to create the files required for indexing

%----------------------------------------------------------------------------------------

\input{commands} % Insert the commands.tex file which contains the majority of the structure behind the template

\begin{document}

%----------------------------------------------------------------------------------------
%	TITLE PAGE
%----------------------------------------------------------------------------------------

\begingroup
\thispagestyle{empty}
\AddToShipoutPicture*{\put(6,5){\includegraphics[scale=1]{background}}} % Image background
\centering
\vspace*{9cm}
\par\normalfont\fontsize{35}{35}\sffamily\selectfont
Building a XMG Compiler\par % Book title
\vspace*{1cm}
{\Huge Simon Petitjean}\par % Author name
\endgroup

%-------------------------------------------------------------------------------%% ---------
%% %	COPYRIGHT PAGE
%% %----------------------------------------------------------------------------------------

%% \newpage
%% ~\vfill
%% \thispagestyle{empty}

%% \noindent Copyright \copyright\ 2013 John Smith\\ % Copyright notice

%% \noindent \textsc{Published by Publisher}\\ % Publisher

%% \noindent \textsc{book-website.com}\\ % URL

%% \noindent Licensed under the Creative Commons Attribution-NonCommercial 3.0 Unported License (the ``License''). You may not use this file except in compliance with the License. You may obtain a copy of the License at \url{http://creativecommons.org/licenses/by-nc/3.0}. Unless required by applicable law or agreed to in writing, software distributed under the License is distributed on an \textsc{``AS IS'' BASIS, WITHOUT WARRANTIES OR CONDITIONS OF ANY KIND}, either express or implied. See the License for the specific language governing permissions and limitations under the License.\\ % License information

%% \noindent \textit{First printing, March 2013} % Printing/edition date

%----------------------------------------------------------------------------------------
%	TABLE OF CONTENTS
%----------------------------------------------------------------------------------------

\chapterimage{chapter_head_1.pdf} % Table of contents heading image

\pagestyle{empty} % No headers

\tableofcontents % Print the table of contents itself

\cleardoublepage % Forces the first chapter to start on an odd page so it's on the right

\pagestyle{fancy} % Print headers again

%----------------------------------------------------------------------------------------
%	CHAPTER 1
%----------------------------------------------------------------------------------------

\chapterimage{chapter_head_3.pdf} % Chapter heading image

\chapter{Writing Language Bricks Files}

\section{Introduction}\index{Introduction}

To build a XMG Compiler that fits your needs, the first step is to define the language you will use. 

The MetaGrammatical language is defined in a modular way, as a combination of elementary parts of language, called Language Bricks. Each language Brick is described into a file, usually named $BrickName.xmg$ 

A Language Brick is a set of grammar rules, that describe all the structures allowed in the language. A grammar rule is the couple of a name (called non terminal symbol) and a structure, which is a sequence of non terminals and terminals (words of the language).\\  

The syntax used to write Language Bricks is very close to the Yacc syntax. %% ref
It begins with the definition of the language terminals and non terminals, and end with the rewriting rules, each of these sections being followed by '$\%\%$' .  

%------------------------------------------------

\section{Terminals}\index{Terminals}

A XMG terminal is basically a keyword of the language. 
The terminal $term$ is declared in the following way:

\begin{theorem}
$\%$token term    
\end{theorem}

\section{Non Terminals}\index{Non Terminals}

The non terminal $NonTerminal$ is declared in the following way:

\begin{theorem}
$\%$type NonTerminal NT    
\end{theorem}

 

\section{Externs}\index{Externs}

There is a special type of non terminals we call Extern non terminals. Their role is to be the connections between Language Bricks. An extern must appear only in the right side of rules.

\begin{theorem}
$\%$ext NonTerminal NT
\end{theorem}

\section{Rules}\index{Rules}

A rule associates a non terminal to the structures it describes. The left part of the rule must be a non terminal, and the right part a sequence of terminals and non teminals.
A rule has the following form:

\begin{theorem}
Node : node Props Feats  ;  
\end{theorem}

The right part of a rule can be a disjunction of sequences:

\begin{theorem}
Node : node Props Feats | node id Props Feats  ;  
\end{theorem}

\section{Punctuation}\index{Punctuation}

Punctuations are parsed in the same way as terminals (they are just special terminals), but do not have to be declared. They come in the right part of rules with simple quotes. Here is an example of rules including punctuation:

\begin{theorem}
Expr : '$\{$' Expr '$\}$' ;  
\end{theorem}

\section{An example Language Brick}

Here is an example of a Language Brick:

\begin{theorem}
\ \\
$\%$type Stmt Stmt\\
$\%$type Call Call\\
$\%$type ids$\_$coma ids$\_$coma\\
$\%$type Dot Dot\\
$\%$type Var Var\\ 
$\%$ext DimStmt DimStmt\\
\\
$\%$token id\\ 
\\
$\%\%$\\
\\
\begin{tabular}{l l}
Stmt : & DimStmt | Stmt ';' Stmt | Stmt '|' Stmt | Var '=' Call | Call \\ 
& | Var '=' Dot | Var '=' Var | '$\{$' Stmt '$\}$' ;\\
Dot : & id '.' Var ;\\
Call : & id '[' ids$\_$coma ']' | id '[' ']';\\
ids$\_$coma : & id ',' ids$\_$coma  | id; \\
Var : & id | '?' id;\\
\end{tabular}
\\
$\%\%$
\end{theorem}



%------------------------------------------------

%% \section{Lists}\index{Lists}

%% Lists are useful to present information in a concise and/or ordered way\footnote{Footnote example...}.

%% \subsection{Numbered List}\index{Lists!Numbered List}

%% \begin{enumerate}
%% \item The first item
%% \item The second item
%% \item The third item
%% \end{enumerate}

%% \subsection{Bullet Points}\index{Lists!Bullet Points}

%% \begin{itemize}
%% \item The first item
%% \item The second item
%% \item The third item
%% \end{itemize}

%% \subsection{Descriptions and Definitions}\index{Lists!Descriptions and Definitions}

%% \begin{description}
%% \item[Name] Description
%% \item[Word] Definition
%% \item[Comment] Elaboration
%% \end{description}

%----------------------------------------------------------------------------------------
%	CHAPTER 2
%----------------------------------------------------------------------------------------

\chapter{Combining Language Bricks}
\section{Creating a Language Brick}\index{Creating a Language Brick}
When all the Language Bricks files needed by the user are written, Language Bricks can be created and combined to build the MetaGrammatical Language. A Language Brick is represented by a python object called BrickGrammar, which takes as a parameter a Language file. 

Every non terminal of the final grammar is the non terminal, as written in the brick, with a prefix indicating in which brick it appears. An optional parameter allows to chose this prefix. Example:  
\begin{theorem}
DeclLang        = BrickGrammar('xmg-mg.xmg',      prefix='XMG')
\end{theorem}

\section{Connecting Bricks}\index{Connecting Bricks}
Every extern non terminal must be connected to the axiom of another Brick. A BrickGrammar object has a function connect, which takes as parameters the extern non-terminal and the Brick to connect. Example:

\begin{theorem}
DeclLang.connect(       'Stmt',    CtrlLang)
\end{theorem}


\section{Generating the Parser}
The parser for the language is fully automatically generated from the Language Bricks.

\chapter{Compiler Bricks}
\section{Patterns}\index{Patterns}

\chapter{Building the Compiler}
\section{Putting it all together}\index{All Together}
\begin{center}
 \includegraphics[scale=0.6]{Pictures/compiler}
\end{center}

%% \chapter{Old Stuff}
%% \section{Theorems}\index{Theorems}

%% This is an example of theorems.

%% \subsection{Several equations}\index{Theorems!Several Equations}

%% \begin{theorem}
%% In $E=\mathbb{R}^n$ all norms are equivalent. It has the properties:
%% \begin{align}
%% & \big| ||\mathbf{x}|| - ||\mathbf{y}|| \big|\leq || \mathbf{x}- \mathbf{y}||\\
%% &  ||\sum_{i=1}^n\mathbf{x}_i||\leq \sum_{i=1}^n||\mathbf{x}_i||\quad\text{where $n$ is a finite integer}
%% \end{align}
%% \end{theorem}

%% \subsection{Single Line}\index{Theorems!Single Line}

%% \begin{theorem}
%% A set $\mathcal{D}(G)$ in dense in $L^2(G)$, $|\cdot|_0$. 
%% \end{theorem}

%% %------------------------------------------------

%% \section{Definitions}\index{Definitions}

%% This is an example of a definition. A definition could be mathematical or it could define a concept.

%% \begin{definition}[Definition name]
%% Given a vector space $E$, a norm on $E$ is an application, denoted $||\cdot||$, $E$ in $\mathbb{R}^+=[0,+\infty[$ such that:
%% \begin{align}
%% & ||\mathbf{x}||=0\ \Rightarrow\ \mathbf{x}=\mathbf{0}\\
%% & ||\lambda \mathbf{x}||=|\lambda|\cdot ||\mathbf{x}||\\
%% & ||\mathbf{x}+\mathbf{y}||\leq ||\mathbf{x}||+||\mathbf{y}||
%% \end{align}
%% \end{definition}

%% %------------------------------------------------

%% \section{Notations}\index{Notations}

%% \begin{notation}
%% Given an open subset $G$ of $\mathbb{R}^n$, the set of functions $\varphi$ are:
%% \begin{enumerate}
%% \item Bounded support $G$;
%% \item Infinitely differentiable;
%% \end{enumerate}
%% a vector space is denoted by $\mathcal{D}(G)$. 
%% \end{notation}

%% %------------------------------------------------

%% \section{Remarks}\index{Remarks}

%% This is an example of a remark.

%% \begin{remark}
%% The concepts presented here are now in conventional employment in mathematics. Vector spaces are taken over the field $\mathbb{K}=\mathbb{R}$, however, established properties are easily extended to $\mathbb{K}=\mathbb{C}$.
%% \end{remark}

%% %------------------------------------------------

%% \section{Corollaries}\index{Corollaries}

%% This is an example of a corollary.

%% \begin{corollary}[Corollary name]
%% The concepts presented here are now in conventional employment in mathematics. Vector spaces are taken over the field $\mathbb{K}=\mathbb{R}$, however, established properties are easily extended to $\mathbb{K}=\mathbb{C}$.
%% \end{corollary}

%% %------------------------------------------------

%% \section{Propositions}\index{Propositions}

%% This is an example of propositions.

%% \subsection{Several equations}\index{Propositions!Several Equations}

%% \begin{proposition}[Proposition name]
%% It has the properties:
%% \begin{align}
%% & \big| ||\mathbf{x}|| - ||\mathbf{y}|| \big|\leq || \mathbf{x}- \mathbf{y}||\\
%% &  ||\sum_{i=1}^n\mathbf{x}_i||\leq \sum_{i=1}^n||\mathbf{x}_i||\quad\text{where $n$ is a finite integer}
%% \end{align}
%% \end{proposition}

%% \subsection{Single Line}\index{Propositions!Single Line}

%% \begin{proposition} 
%% Let $f,g\in L^2(G)$; if $\forall \varphi\in\mathcal{D}(G)$, $(f,\varphi)_0=(g,\varphi)_0$ then $f = g$. 
%% \end{proposition}

%% %------------------------------------------------

%% \section{Examples}\index{Examples}

%% This is an example of examples.

%% \subsection{Equation and Text}\index{Examples!Equation and Text}

%% \begin{example}
%% Let $G=\{x\in\mathbb{R}^2:|x|<3\}$ and denoted by: $x^0=(1,1)$; consider the function:
%% \begin{equation}
%% f(x)=\left\{\begin{aligned} & \mathrm{e}^{|x|} & & \text{si $|x-x^0|\leq 1/2$}\\
%% & 0 & & \text{si $|x-x^0|> 1/2$}\end{aligned}\right.
%% \end{equation}
%% The function $f$ has bounded support, we can take $A=\{x\in\mathbb{R}^2:|x-x^0|\leq 1/2+\epsilon\}$ for all $\epsilon\in\intoo{0}{5/2-\sqrt{2}}$.
%% \end{example}

%% \subsection{Paragraph of Text}\index{Examples!Paragraph of Text}

%% \begin{example}[Example name]
%% \lipsum[2]
%% \end{example}

%% %------------------------------------------------

%% \section{Exercises}\index{Exercises}

%% This is an example of an exercise.

%% \begin{exercise}
%% This is a good place to ask a question to test learning progress or further cement ideas into students' minds.
%% \end{exercise}

%% %------------------------------------------------

%% \section{Problems}\index{Problems}

%% \begin{problem}
%% What is the average airspeed velocity of an unladen swallow?
%% \end{problem}

%% %------------------------------------------------

%% \section{Vocabulary}\index{Vocabulary}

%% Define a word to improve a students' vocabulary.

%% \begin{vocabulary}[Word]
%% Definition of word.
%% \end{vocabulary}

%% %----------------------------------------------------------------------------------------
%% %	CHAPTER 3
%% %----------------------------------------------------------------------------------------

%% \chapterimage{chapter_head_1.pdf} % Chapter heading image

%% \chapter{Presenting Information}

%% \section{Table}\index{Table}

%% \begin{table}[h]
%% \centering
%% \begin{tabular}{l l l}
%% \toprule
%% \textbf{Treatments} & \textbf{Response 1} & \textbf{Response 2}\\
%% \midrule
%% Treatment 1 & 0.0003262 & 0.562 \\
%% Treatment 2 & 0.0015681 & 0.910 \\
%% Treatment 3 & 0.0009271 & 0.296 \\
%% \bottomrule
%% \end{tabular}
%% \caption{Table caption}
%% \end{table}

%% %------------------------------------------------

%% \section{Figure}\index{Figure}

%% \begin{figure}[h]
%% \centering\includegraphics[scale=0.5]{placeholder}
%% \caption{Figure caption}
%% \end{figure}

%% %----------------------------------------------------------------------------------------
%% %	BIBLIOGRAPHY
%% %----------------------------------------------------------------------------------------

\chapter*{Bibliography}
\addcontentsline{toc}{chapter}{\textcolor{ocre}{Bibliography}}
\section*{Books}
\addcontentsline{toc}{section}{Books}
\printbibliography[heading=bibempty,type=book]
\section*{Articles}
\addcontentsline{toc}{section}{Articles}
\printbibliography[heading=bibempty,type=article]

%----------------------------------------------------------------------------------------
%	INDEX
%----------------------------------------------------------------------------------------

\cleardoublepage
\setlength{\columnsep}{0.75cm}
\addcontentsline{toc}{chapter}{\textcolor{ocre}{Index}}
\printindex

%----------------------------------------------------------------------------------------

\end{document}
